\begin{evenBlock}{Triangle Passing}
\textbf{Drill Description:}
        This drill is like the `4 Corner Passing Drill' but incorporates player movement to insure the player with the ball always has two options to pass too.
        If the groups are uneven, a `defender' can be added to the box.  If the pass goes through the box the passer switches location with the defender.  If trap is on the wrong foot the trapper switches with the defender.  Defenders count 5 successful passes and they switch with a player.
\begin{minipage}[t]{\linewidth}
    \begin{minipage}{.3\linewidth} % Left column and width
        \centering
        \includegraphics[width=\textwidth]{../img/Trimmed/Triangle_Passing_3P_mini}
    \end{minipage}
    \hspace{0.05\linewidth}
    \begin{minipage}{.6\linewidth} % Left column and width
        \begin{enumerate}
            \setlength{\itemsep}{0pt}
            \setlength{\parskip}{0pt}
            \setlength{\parsep}{0pt}
            \item All players stand `open' so they can see the other two players.
            \item The ball should be passed on one direction to start (to the left is more natural for a right footed player).
            \item The player receiving the ball should move his body so he receives the ball on his left foot then passes it to the next player using his right.  However the pass should wait until the 3 player is in position.
            \item  Player 3 (P3) was at a corner nearest the ball, however once the ball was passed, P3 needs to move to the other corner so they are again at a corner adjacent to the player with the ball.
            \item After 5 rounds around the box, switch directions.  Pass to the right using the left foot, trap with the right.
            \item After 5 additional rounds allow the player to switch directions at will, but any two adjacent players can't pass the ball back and forth more than 3 times.
        \end{enumerate}
    \end{minipage}
\end{minipage}
\raggedright
    \textbf{Coaching Points:}
    \begin{itemize}
        \setlength{\itemsep}{0pt}
        \setlength{\parskip}{0pt}
        \setlength{\parsep}{0pt}
        \item Explain the first touch with the correct foot is the most important part.
        \item The touch should place the ball one step away from the player so they can step into and make a strong pass.
        \item The goal is to use two touches, not 1 and not 3.
    \end{itemize}

\end{evenBlock}